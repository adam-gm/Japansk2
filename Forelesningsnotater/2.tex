\section{Week 4}
\subsection{L8 元気}

い adjective long form : かわいいです

short form: かわいい without です

な adjectives and nouns long form: しずかです

short form: しずかだ

Present negative: ない

ここでしゃしんをとらないでください - politer than ここでしゃしんをとってはいけません

おもう - 思います : to think

と 思います : to think that

しつもん: 今日はさむいとおもうか

こたえ A: ちょっとさむいと思います : I think that it is a little cold.

こたえ B: さむくないと思います : I think that it is not cold.

いう/言う : to say

と言っていました : said that (ていました form, something that was ongoing)

\subsection{の - verb}
Adding の after a verb makes the action as a noun. This makes the verb rather act as a subject/object in the sentence.
For example: I like shopping.

ひるごはんを食べるのをわすれます - Forgot eating lunch

かいものするのがすきだ - Like to do shopping

しつもん: 明日も大学にきますか

Can drop か, but need to use correct intonation to make it obvious its a question. Go abit higher at the end saying くる.

Example: 明日も大学にくる

こたえ: はいきます/くる or いいえきません/こない

うん/ううん - Yes/No say it "inside" mouth, not actually pronouncing "un".

Conversation:
ぼくのうちにいぬがいます

しつもん: そのいぬはかわいい?

こたえ: うん、かわいいよ or ううん、あまりかわいくない.

元気な - na adjective

しつもん: お元気ですか

こたえ: 元気だよ (for men should add da+yo, not only yo) or 元気だね (ね when talking specific to a person and "agreeing")

学生だよ or 学生じゃないよ : Im a student "you know" vs Im not a student ..

\subsection{が particle}
X がすきです / きらいです










