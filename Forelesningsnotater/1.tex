\section{Week 3}
\subsection{て-form}
ペ-ジ 127

To live: すむ/すみます、て-form: すんでいます

\begin{itemize}
    \item えいがを見る、 Say this when action starts。
    \item えいがを見ています、Say this when aciton keeps going on。
    \item えいがを見ていません、見ました、見た、Say this when action is over。
    \item 見た: Short form past tense.
\end{itemize}


ぼくは今トロンハイムにすんでいます - I live in Trondheim now。

今日ごぜん五時ごろおきていましたか、ねていましたか. This morning around 5 o'clock where you awake/sleeping?

しぬ/しにます - to die -> しんでいる。
いきる/いきます - to be alive -> 生きている。

しる/しります - to know/to be aware of -> しっています。
Dont say しっていません、say しりません。
Reasoning from さちこ先生: Once you learn A, your mind cant return to the state which you dont know A. Therefore,
you dont say, “しっていません”. Only when you havent learned A, you dont know A. Therefore,
the negative is “しりません”

おしごとは何ですか - What is your job/work?

ぼくのせんこうはサイバネティクスです。My main study subject is cybernetics.

車/くるま、Car. 車をもつ/もちます -> To own a car.

お金がありません -> I dont have money.

すわる -> to sit. すわっています -> sitting

行く、行って
くる、きて
かえる、かえって

ばんごはんを食べにかえります。Im coming home to eat dinner.

\subsection{Adjectives}
\subsubsection{い - adjective}
くらい -> Dark/gloomy
今日はくらくてさむいです -> Today it is dark and cold.

\subsubsection{な - adjective}
きれいな、て-form: きれいで
しずかな、て-form: しずかで

きれいでよかったです -> It was beautiful and pleasant.

\subsection{Self introduction}
NTNUの学生です。せんこうはサイバネティクスです。NTNUに行っています。
NTNUに行っています-> Can mean going to NTNU as actually going to the uni, or as in being a student there.





